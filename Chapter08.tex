%%% Advanced Calculus
%%% Chapter 8: Continuous Functions in Euclidean Spaces
%%% Manuscript pages 33-60
%%% Spring 2019
%%% Typeset by Ecem Ilgun and Ata Deniz Aydin
\documentclass[12pt]{amsbook}
\usepackage[mathscr]{eucal}
\usepackage{amsmath,amsfonts}

%%% Set the parskip
\parskip=\smallskipamount
%%% Set the page dimensions
\hoffset -1.5cm \voffset -1cm \textwidth 15.5truecm
\textheight21.5truecm
%%% Define theorems and the like
\newtheorem{theorem}{Theorem}[section]
\newtheorem{fact}[theorem]{Fact}
\newtheorem{proposition}[theorem]{Proposition}
\newtheorem{corollary}[theorem]{Corollary}
\newtheorem{lemma}[theorem]{Lemma}
%%% Define definitions the like
\theoremstyle{definition}
\newtheorem{definition}[theorem]{Definition}
\newtheorem{example}[theorem]{Example}
\newtheorem{remark}[theorem]{Remark}
%%% The following paragraph writes the equation numbers with two counters,
%%% the first is the section number and the second resets within the section.

\makeatletter
\@addtoreset{equation}{section}
\makeatother
\renewcommand\theequation{\arabic{section}.\arabic{equation}}

%%% The following paragraph sets macros for mathematical bold letters

\newcommand{\CC}{{\mathbb C}}
\newcommand{\NN}{{\mathbb N}}
\newcommand{\QQ}{{\mathbb Q}}
\newcommand{\ZZ}{{\mathbb Z}}
\newcommand{\DD}{{\mathbb D}}
\newcommand{\RR}{{\mathbb R}}

%%% The following paragraph sets macros for mathematical caligraphic letters


\newcommand{\cA}{{\mathcal A}}
\newcommand{\cB}{{\mathcal B}}
\newcommand{\cC}{{\mathcal C}}
\newcommand{\cD}{{\mathcal D}}
\newcommand{\cE}{{\mathcal E}}
\newcommand{\cF}{{\mathcal F}}
\newcommand{\cG}{{\mathcal G}}
\newcommand{\cH}{{\mathcal H}}
\newcommand{\cJ}{{\mathcal J}}
\newcommand{\cK}{{\mathcal K}}
\newcommand{\cL}{{\mathcal L}}
\newcommand{\cM}{{\mathcal M}}
\newcommand{\cN}{{\mathcal N}}
\newcommand{\cP}{{\mathcal P}}
\newcommand{\cR}{{\mathcal R}}
\newcommand{\cS}{{\mathcal S}}
\newcommand{\cT}{{\mathcal T}}
\newcommand{\cU}{{\mathcal U}}
\newcommand{\cV}{{\mathcal V}}
\newcommand{\cW}{{\mathcal W}}
\newcommand{\cZ}{{\mathcal Z}}

%%% Some macros

\newcommand{\Ra}{\Rightarrow} % Double right arrow, for implications
\newcommand{\La}{\Leftarrow} % Double left arrow, for implications
\newcommand{\Lra}{\Leftrightarrow} % Double equivalence sign
\newcommand{\ran}{\mathrm{Ran}} % Range
\newcommand{\ra}{\rightarrow} % Simple right arrow
\newcommand{\ol}{\overline} % Line over symbols
\newcommand{\hra}{\hookrightarrow} % Hooked right arrow
\newcommand{\de}{\mathrm{d}} % For differentials, integrals, etc.

%%% The beginning of the document
\begin{document}

\chapter{Continuous Functions in Euclidean Spaces}
%% AC02-3 pp. 33-42

\section{Continuous Functions}

% AC02 p. 33

\begin{definition} 
Let $(X; \rho)$ and $(Y; \eta)$ be two metric spaces, $f: X \ra Y$ and $x_0 \in X$.

$\bullet$ $f$ is \emph{continuous} at $x_0$ if $\forall \epsilon > 0 \ \exists \delta > 0$ such that $\forall x \in X$, if $\rho(x, x_0) < \delta$ then $\eta(f(x), f(x_0) < \epsilon$

$\bullet$ $f$ is \emph{continuous} if $f$ is continuous at all $x_0 \in X$.
\end{definition}

\begin{remark}
We may consider the "more general" setting \begin{equation*} f: D \ (\subseteq X) \ra Y, \ x_0 \in D. \end{equation*}

$f$ is \emph{continuous} at $x_0$ if $\forall \epsilon > 0 \ \exists \delta > 0$ such that $\forall x \in D$, if $\rho(x, x_0) < \delta$ then $\eta(f(x), f(x_0) < \epsilon$. 

But this is not more general than the definition since it coincides with the case when $D$ is considered as a metric space with the metric $\rho|_{D \times D}$.
\end{remark}

\begin{fact}
Let $(X; \rho)$ and $(Y; \eta)$ be two metric spaces, $f: X \ra Y$ and $x_0 \in X$. TFAE:

\begin{itemize}
\item[(i)] $f$ is continuous at $x_0$
\item[(ii)] $\forall \epsilon > 0 \ \exists \delta > 0$ such that $f(B_\delta(x_0)) \subseteq B_\epsilon(f(x_0))$
\item[(iii)] $\forall U \in \cV(f(x_0)) \ \exists V \in \cV(x_0)$ such that $f(V) \subseteq U$.
\item[(iv)] $\forall U$ open in Y such that $f(x_0) \in U$ $\exists V$ open in X such that $x_0 \in V$ and $f(V) \subseteq U$.
\end{itemize}
\end{fact}

The last characterization shows that continuity is a topological concept.

% AC02 p. 34

\begin{definition}
Let $(X; \cT)$ and $(Y; \mathcal{Y})$ be topological spaces. Let $f: X \ra Y$ be a function and $x_0 \in X$. 

$f$ is \emph{continuous} at $x_0$ if $\forall U \in \mathcal{Y}$ such that $f(x_0) \in U$ $\exists V \in \cT$ such that $x_0 \in V$ and $f(V) \subseteq U$.

\end{definition} 

Continuity is related to the more general concept of limit.

\begin{definition}
Let $(X; \rho)$ and $(Y; \eta)$ be topological spaces, $D \subseteq X$. Let $f: D \ra Y$ be a function and $x_0 \in X, y_0 \in Y$. 

$f$ \emph{has limit} $y_0$ \emph{at} $x_0$ if:

$\bullet$ $x_0$ is an accumulation point for $D$

$\bullet$ $\forall \epsilon > 0$ $\exists \delta > 0$ such that $\forall x \in D \setminus \{x_0\}$, if $\rho(x, x_0) < \delta$ then $\eta(f(x), y_0) < \epsilon$.

\end{definition} 

\begin{remark} 
If $f$ has limit $y_0$ at $x_0$ then $y_0$ is unique, hence we can denote: \begin{equation*}\lim_{x \ra x_0} f(x) = y_0.\end{equation*}
\end{remark}

% p. 35

\begin{fact} \
\begin{itemize}
\item[(1)] $f: D \left(\subseteq X\right) \ra Y$, metric spaces, $x_0 \in X$. % Domain of f labelled as X in notes, should be D for isolated pts. etc. to make sense
    \begin{itemize}
    \item[(i)] if $x_0$ is isolated in $D$ then $f$ is continuous at $x_0$
    \item[(ii)] if $x_0$ is an accum. point for $D$ then $f$ is cont. at $x_0$ iff \begin{equation*}\lim_{x \ra x_0} f(x) = f(x_0).\end{equation*}
    \end{itemize}
\item[(2)] % crossed out
\item[(3)] $f: D \ra Y$ function, $x_0 \in D$ accumulation point and $y_0 \in Y$. Then
    \begin{align}
    \lim_{x \ra x_0} f(x) = y_0 \ \Lra \ \nonumber
    & \forall (x_n)_{n \geq 1} \textit{ with all } x_n \in D \\ \nonumber
    & \textit{and } x_n \neq x_0 \ \forall n \\ \nonumber
    & \textit{and } \lim_{n \ra \infty} x_n = x_0 \\ \nonumber
    & \textit{we have } \lim_{n \ra \infty} f(x_n) = y_0.
    \end{align}
    \end{itemize}
    
    \begin{proof}
    
    \begin{itemize}
    \item["$\Ra$"] $\forall \epsilon > 0$ $\exists \delta > 0$ such that $\forall x \in D \setminus \{x_0\}$, if $\rho(x, x_0) < \delta$ then $\eta(f(x), y_0) < \epsilon$.
    
    Take $(x_n)_{n \geq 1}$ a sequence in $X$, $x_n \neq x_0 \ \forall n$ and $\lim_{n \ra \infty} x_n = x_0$.
    
    Then $\exists N_\delta \in \NN$ such that $\forall n \in \NN$, if $n \geq N_\delta$ then $\rho(x_n, x_0) < \delta$ hence: $\eta(f(x_n), y_0) < \epsilon$.
    
    % p. 36
    
    \item["$\La$"] Assume that $\forall (x_n)_{n \geq 1}$ seq. with all elements in $D$ such that $x_n \neq x_0 \ \forall n \in \NN$
    and $\lim_{n \ra \infty} x_n = x_0$ we have $\lim_{n \ra \infty} f(x_n) = y_0$.
    
    By contradiction assume that $f(x)$ does not converge to $y_0$ as $x$ approaches $x_0$. 
    
    Then $\exists \epsilon_0 > 0$ such that $\forall \delta > 0$ $\exists x \in D \setminus \{x_0\}$ with $\rho(x, x_0) < \delta$ and $\eta(f(x), y_0) \geq \epsilon_0$.
    
    $\forall n \in \NN$, take $\delta = \frac{1}{n} > 0$ hence $\exists x_n \in D \setminus \{x_0\}$ with $\rho(x_n, x_0) < \delta = \frac{1}{n}$ and $\eta(f(x_n), y_0) \geq \epsilon_0$
    
    hence $x_n \xrightarrow[n \ra \infty]{\rho} x_0$ but $f(x_n) \not\xrightarrow[n \ra \infty]{\eta} y_0$. % how to cross out arrow better?
    \end{itemize}
    \end{proof}

\item[(4)] (Sequential Characterization of Continuity) Let $f: X \ra Y$ function and $x_0 \in X$. 
\begin{align}
\textit{Then } f \textit{ is continuous at } x_0 \Lra \ \nonumber
& \forall (x_n)_{n \geq 1} \textit{ seq. in } X \\ \nonumber
& \textit{such that } x_n \xrightarrow[n]{\rho} x_0 \\ \nonumber
& \textit{we have } f(x_n) \xrightarrow[n]{\eta} y_0 .
\end{align}

% p. 37

\item[(5)] (Composition of Functions) Let $\underset{\rho}X \overset{f}\ra \underset{\eta}Y \overset{g}\ra \underset{\theta}Z$ functions between metric spaces.
    \begin{itemize}
    \item[(i)] % crossed out
    \item[(ii)] \begin{align}
    \textit{Assume that } \nonumber
    & x_0 \in X \\ \nonumber
    & f \textit{ is cont. at } x_0 \\ \nonumber
    & g \textit{ is cont. at } f(x_0)
    \end{align}
    Then $g \circ f$ is continuous at $x_0$.
    \end{itemize}

\item[(6)] (Functions Between Euclidean Spaces) Let $D \subseteq \RR^p$ and $f: D \ra \RR^q$ be a function, hence $f(x) = \left(f_1(x), \ldots, f_q(x)\right)$, $\forall x \in D$ where  $f_j : D \ra \RR$ function $\forall j = 1, \ldots, q$.
    \begin{itemize}
    \item[(i)] Let $x_0 \in D^\prime$, i.e. $x_0$ is an accumulation point for $D$ and \\ $y^{(0)} = \left(y^{(0)}_1, \ldots, y^{(0)}_q\right) \in \RR^q$. Then
    \begin{equation*}
    \lim_{x \ra x_0} f(x) = y^{(0)} \Lra \ \forall j = 1, \ldots, q \  \lim_{x \ra x_0} f_j(x) = y^{(0)}_j.
    \end{equation*}
    % p. 38
    \item[(ii)] Let $x_0 \in D$. Then
    \begin{equation*}
    f \textit{ is cont. at } x_0 \Lra \ \forall j = 1, \ldots, q \ , f_j \textit{ is cont. at } x_0.
    \end{equation*}
    \end{itemize}
    
    \begin{proof} \
    \begin{itemize}
    \item[(i)] \
    	\begin{itemize}
	\item["$\Ra$".] Assume that $\lim_{x \ra x_0} f(x) = y^{(0)}$ and use the $\|\cdot\|_\infty$.
	
	$\forall \epsilon > 0 \ \exists \delta > 0$ such that $\forall x \in D \setminus \{x_0\}$, if $\|x - x_0\| < \delta$ then $\|f(x) - y^{(0)}\|_\infty < \epsilon$.
	
	Let $j \in \{1, \ldots, q\}$, then \begin{equation*} |f_j(x) - y^{(0)}_j| \leq \|f(x) - y^{(0)}\|_\infty < \epsilon. \end{equation*}
	
	\item["$\La$".] Assume that $\forall j = 1, \ldots, q$ \begin{equation*} \lim_{x \ra x_0} f_j(x) = y^{(0)}_j. \end{equation*}
	
	Then $\forall \epsilon > 0$ $\exists \delta_j > 0$ such that $\forall x \in D \setminus \{x_0\}$, if $\|x-x_0\|_\infty < \delta_j$ then $|f_j(x) - y^{(0)}_j| < \epsilon$. 
	
	Take $\delta := \min\{\delta_1, \ldots, \delta_q\} > 0$.
	
	Then $\forall j = 1, \ldots, q$, if $\|x-x_0\|_\infty < \delta \leq \delta_j$ then $|f_j(x) - y^{(0)}_j| < \epsilon$ hence \begin{equation*} \|f(x) - y^{(0)}\|_\infty = \max_{j=1}^q \{|f_j(x) - y^{(0)}_j|\} < \epsilon. \end{equation*}
	\end{itemize}
    \item[(ii)]
    $\bullet$ if $x_0$ isolated, nothing to prove.
    
    $\bullet$ if $x_0$ accum. point for D, we use (i). % align these later
    \end{itemize}
    \end{proof}

\end{fact}

% p. 39

\section{Continuity and Topology}
%% AC03 pp. 43-44

\section{Continuity and Compactness}
%% AC03 pp. 45-46

\section{Continuity and Connectedness}
%% AC03 pp. 47-50

\section{Uniform Continuity}
%% AC03 pp. 51-54

\section{Sequences of Functions}
%% AC03 pp. 55-59

\section{Series of Functions}
%% AC03 p. 60

\end{document}
