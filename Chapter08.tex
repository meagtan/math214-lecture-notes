%%% Advanced Calculus
%%% Chapter 8: Continuous Functions in Euclidean Spaces
%%% Manuscript pages 33-60
%%% Spring 2019
%%% Typeset by Ecem Ilgun and Ata Deniz Aydin
\documentclass[12pt]{amsbook}
\usepackage[mathscr]{eucal}
\usepackage{amsmath,amsfonts}

%%% Set the parskip
\parskip=\smallskipamount
%%% Set the page dimensions
\hoffset -1.5cm \voffset -1cm \textwidth 15.5truecm
\textheight21.5truecm
%%% Define theorems and the like
\newtheorem{theorem}{Theorem}[section]
\newtheorem{fact}[theorem]{Fact}
\newtheorem{proposition}[theorem]{Proposition}
\newtheorem{corollary}[theorem]{Corollary}
\newtheorem{lemma}[theorem]{Lemma}
%%% Define definitions the like
\theoremstyle{definition}
\newtheorem{definition}[theorem]{Definition}
\newtheorem{example}[theorem]{Example}
\newtheorem{remark}[theorem]{Remark}
%%% The following paragraph writes the equation numbers with two counters,
%%% the first is the section number and the second resets within the section.

\makeatletter
\@addtoreset{equation}{section}
\makeatother
\renewcommand\theequation{\arabic{section}.\arabic{equation}}

%%% The following paragraph sets macros for mathematical bold letters

\newcommand{\CC}{{\mathbb C}}
\newcommand{\NN}{{\mathbb N}}
\newcommand{\QQ}{{\mathbb Q}}
\newcommand{\ZZ}{{\mathbb Z}}
\newcommand{\DD}{{\mathbb D}}
\newcommand{\RR}{{\mathbb R}}

%%% The following paragraph sets macros for mathematical caligraphic letters


\newcommand{\cA}{{\mathcal A}}
\newcommand{\cB}{{\mathcal B}}
\newcommand{\cC}{{\mathcal C}}
\newcommand{\cD}{{\mathcal D}}
\newcommand{\cE}{{\mathcal E}}
\newcommand{\cF}{{\mathcal F}}
\newcommand{\cG}{{\mathcal G}}
\newcommand{\cH}{{\mathcal H}}
\newcommand{\cJ}{{\mathcal J}}
\newcommand{\cK}{{\mathcal K}}
\newcommand{\cL}{{\mathcal L}}
\newcommand{\cM}{{\mathcal M}}
\newcommand{\cN}{{\mathcal N}}
\newcommand{\cP}{{\mathcal P}}
\newcommand{\cR}{{\mathcal R}}
\newcommand{\cS}{{\mathcal S}}
\newcommand{\cT}{{\mathcal T}}
\newcommand{\cU}{{\mathcal U}}
\newcommand{\cV}{{\mathcal V}}
\newcommand{\cW}{{\mathcal W}}
\newcommand{\cZ}{{\mathcal Z}}

%%% Some macros

\newcommand{\Ra}{\Rightarrow} % Double right arrow, for implications
\newcommand{\La}{\Leftarrow} % Double left arrow, for implications
\newcommand{\Lra}{\Leftrightarrow} % Double equivalence sign
\newcommand{\ran}{\mathrm{Ran}} % Range
\newcommand{\ra}{\rightarrow} % Simple right arrow
\newcommand{\ol}{\overline} % Line over symbols
\newcommand{\hra}{\hookrightarrow} % Hooked right arrow
\newcommand{\de}{\mathrm{d}} % For differentials, integrals, etc.

%%% The beginning of the document
\begin{document}

\chapter{Continuous Functions in Euclidean Spaces}
%% AC02-3 pp. 33-42

\section{Continuous Functions}

\begin{definition} 
Let $(X; \rho)$ and $(Y; \eta)$ be two metric spaces, $f: X \ra Y$ and $x_0 \in X$.

$\bullet$ $f$ is \emph{continuous} at $x_0$ if $\forall \epsilon > 0 \ \exists \delta > 0$ such that $\forall x \in X$, if $\rho(x, x_0) < \delta$ then $\eta(f(x), f(x_0) < \epsilon$

$\bullet$ $f$ is \emph{continuous} if $f$ is continuous at all $x_0 \in X$.
\end{definition}



\begin{remark}
We may consider the "more general" setting $$f: D \ (\subseteq X) \ra Y, \ x_0 \in D.$$

$f$ is \emph{continuous} at $x_0$ if $\forall \epsilon > 0 \ \exists \delta > 0$ such that $\forall x \in D$, if $\rho(x, x_0) < \delta$ then $\eta(f(x), f(x_0) < \epsilon$. 

But this is not more general than the definition since it coincides with the case when $D$ is considered as a metric space with the metric $\rho|_{D \times D}$.
\end{remark}



\begin{fact}
Let $(X; \rho)$ and $(Y; \eta)$ be two metric spaces, $f: X \ra Y$ and $x_0 \in X$. TFAE:

\begin{itemize}
\item[(i)] $f$ is continuous at $x_0$
\item[(ii)] $\forall \epsilon > 0 \ \exists \delta > 0$ such that $f(B_\delta(x_0)) \subseteq B_\epsilon(f(x_0))$
\item[(iii)] $\forall U \in \cV(f(x_0)) \ \exists V \in \cV(x_0)$ such that $f(V) \subseteq U$.
\item[(iv)] $\forall U$ open in Y such that $f(x_0) \in U$ $\exists V$ open in X such that $x_0 \in V$ and $f(V) \subseteq U$.
\end{itemize}

The last characterization shows that continuity is a topological concept.
\end{fact}

\section{Continuity and Topology}
%% AC03 pp. 43-44

\section{Continuity and Compactness}
%% AC03 pp. 45-46

\section{Continuity and Connectedness}
%% AC03 pp. 47-50

\section{Uniform Continuity}
%% AC03 pp. 51-54

\section{Sequences of Functions}
%% AC03 pp. 55-59

\section{Series of Functions}
%% AC03 p. 60

\end{document}
