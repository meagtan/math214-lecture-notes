%%% Advanced Calculus
%%% An example file
%%% 
%%% 2019/02/16
%%% 
\documentclass[12pt]{amsbook}
\usepackage[mathscr]{eucal}
\usepackage{amsmath,amsfonts}
\usepackage{amsthm}

%%% Set the parskip
\parskip=\smallskipamount
%%% Set the page dimensions
\hoffset -1.5cm \voffset -1cm \textwidth 15.5truecm
\textheight21.5truecm
%%% Define theorems and the like
\newtheorem{theorem}{Theorem}[section]
\newtheorem{fact}[theorem]{Fact}
\newtheorem{proposition}[theorem]{Proposition}
\newtheorem{corollary}[theorem]{Corollary}
\newtheorem{lemma}[theorem]{Lemma}
%%% Define definitions and the like
\theoremstyle{definition}
\newtheorem{definition}[theorem]{Definition}
\newtheorem{example}[theorem]{Example}
\newtheorem{remark}[theorem]{Remark}

%%% The following paragraph writes the equation numbers with two counters,
%%% the first is the section number and the second resets within the section.

\makeatletter
\@addtoreset{equation}{section}
\makeatother
\renewcommand\theequation{\arabic{section}.\arabic{equation}}

%%% The following paragraph sets macros for mathematical bold letters

\newcommand{\CC}{{\mathbb C}}
\newcommand{\NN}{{\mathbb N}}
\newcommand{\QQ}{{\mathbb Q}}
\newcommand{\ZZ}{{\mathbb Z}}
\newcommand{\DD}{{\mathbb D}}
\newcommand{\RR}{{\mathbb R}}

%%% The following paragraph sets macros for mathematical caligraphic letters


\newcommand{\cA}{{\mathcal A}}
\newcommand{\cB}{{\mathcal B}}
\newcommand{\cC}{{\mathcal C}}
\newcommand{\cD}{{\mathcal D}}
\newcommand{\cE}{{\mathcal E}}
\newcommand{\cF}{{\mathcal F}}
\newcommand{\cG}{{\mathcal G}}
\newcommand{\cH}{{\mathcal H}}
\newcommand{\cJ}{{\mathcal J}}
\newcommand{\cK}{{\mathcal K}}
\newcommand{\cL}{{\mathcal L}}
\newcommand{\cM}{{\mathcal M}}
\newcommand{\cN}{{\mathcal N}}
\newcommand{\cP}{{\mathcal P}}
\newcommand{\cR}{{\mathcal R}}
\newcommand{\cS}{{\mathcal S}}
\newcommand{\cT}{{\mathcal T}}
\newcommand{\cU}{{\mathcal U}}
\newcommand{\cV}{{\mathcal V}}
\newcommand{\cW}{{\mathcal W}}
\newcommand{\cZ}{{\mathcal Z}}

%%% Some macros

\newcommand{\Ra}{\Rightarrow} % Double right arrow, for implications
\newcommand{\La}{\Leftarrow} % Double left arrow, for implications
\newcommand{\Lra}{\Leftrightarrow} % Double equivalence sign
\newcommand{\ran}{\mathrm{Ran}} % Range
\newcommand{\ra}{\rightarrow} % Simple right arrow
\newcommand{\ol}{\overline} % Line over symbols
\newcommand{\hra}{\hookrightarrow} % Hooked right arrow
\newcommand{\de}{\mathrm{d}} % For differentials, integrals, etc.

%%% The beginning of the document
\begin{document}

\chapter{This is the Title of the Chapter}

\section{This is the Title of the Section}


\begin{theorem} This is a theorem.
\end{theorem}

\begin{proof}
This is a proof. Sometimes a proof is divided in steps, which in the
manuscript are indicated by bullets.

$\bullet$ \emph{Here is the statement that we want to prove as the first
  step. Note that it is written in italics by using the emphasizing
  environment.}

Now the argument for the first step follows.

$\bullet$ \emph{Here is the statement that we want to prove as the second
  step. It is also written in italics by using the emphasizing
  environment.}

Now the argument for the second step follows. 

Sometimes we need displayed formulas, as follows:
\begin{equation*}
f(t)=\begin{cases} t, & \mbox{ if }t>0,\\ -t, & \mbox{ if }t\leq 0.
\end{cases}
\end{equation*}

If there is a chain of displayed formulas, use the following environment
\begin{align*}
f(t) & = \frac{1}{1+t^2}+\int_0^t g(s)\de s+ \frac{\de f}{\de t} \\
& = \frac{1}{1+t^2}+ \log(1+t^2)\exp(2\pi t)+ \frac{\de f}{\de t}.
\end{align*}
You may observe how the $\de f$, the symbol for differential, is used by using
the provided macro. 

If there is a matrix you can do it in the following way:
\begin{equation*}
A=\left[\begin{matrix} 1 & 2 \\ 1 & 3 \end{matrix}\right].
\end{equation*}

Note the square that indicates the end of a proof.
\end{proof}

\begin{definition} This is a definition. Note that it appears in roman
  letters. Only the \emph{defined term} should be emphasized (in the
  manuscript it is underlined).
\end{definition}

\subsection{This is a subsection}

\begin{fact} This is a fact, something that I did not decide whether to be a
  theorem, proposition, or something else. Sometimes, we need to have a list
  of
conditions, that we can do as follows:
\begin{itemize}
\item[(i)] First condition.
\item[(ii)] Second condition.
\end{itemize}
\end{fact}

\begin{proof} Here comes the proof of the Fact. If the corresponding fact does
  not have a proof, this part is not used at all.

In case a displayed formula has to be cited, you may use labels and numbered
equations:
\begin{equation}\label{e:mylabel} f(t)=f(t+1),\quad\mbox{ for all }t\in\RR.
\end{equation}
Then you can use the label to cite it as \eqref{e:mylabel}.

In case there is a chain of equations, and only some of them have to be
labelled, you do like that
\begin{align}
f(t) & = \frac{1}{1+t^2}+\int_0^t g(s)\de s+ \frac{\de f}{\de t} \nonumber \\
& = \frac{1}{1+t^2}+ \log(1+t^2)\exp(2\pi t)+ \frac{\de f}{\de
  t}.\label{e:label} 
\end{align}

In case a proof ends with a displayed formula, you should indicate the place
of the square that marks the end of the proof:
\begin{equation*}
f(t) =g (t).\qedhere
\end{equation*}
\end{proof}
\end{document}
